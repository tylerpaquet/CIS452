\documentclass[11pt]{article}
\usepackage{palatino}
\usepackage{fullpage}
\usepackage{listings}
%\usepackage{minted}

% Shell command markup
\newcommand{\shellcmd}[1]{\texttt{\footnotesize #1}}

% Rename this file to cis452-lectureNN.tex (use 2-digit consecutive
% numbers)
% To produce a PDF output, type the following on EOS
%
%  pdflatex -shell-escape cis452-lecture??.tex
%

\title{CIS452 Lecture-40 Notes}
\author{Winter 2017}
\date{}
\begin{document}
\maketitle

\noindent

% Be sure to replace your name below
\textbf{Scribe:} \textit{Tyler Paquet}

\section*{Announcements}

% Use itemize for bullet list
\begin{itemize}
  \item Continued to look through handout on Block Allocation
\end{itemize}

\section*{Chapter 11 (Continued)}

% Use enumerate for numbered list

\begin{enumerate}
\item Indexed Allocation: Limitations
	\begin{itemize}
  	\item The maximum file size if the number of pointers that can fit into one index block
  	\item To store large files
		\begin{itemize}
  		\item Link several index blocks together
  		\item Multi-level index
  		\item Combined: use both direct index and multilevel indices (Unix File Systems)
		\end{itemize}  
	\end{itemize}
	
\item UFS File Format
	\begin{itemize}
  	\item UFS (Unix File System) refers a set of FS variants based on Unix design
  	\item Boot block (usually a single sector): code to bootstrap the OS
  	\item Super block (several disk sectors): metadata about the filesystem
  	\item Inode block (several disk sectors) for storing index nodes (inodes)
  		\begin{itemize}
  		\item One inode per file/directory
  		\end{itemize}
  	\item Many data blocks (greater than 95 percent of the disk sectors)
  		\begin{itemize}
  		\item Directory entry and File objects
  		\item Journals used during recovery
  		\end{itemize}
	\end{itemize}
	
\item UFS Super Block
	\begin{itemize}
	\item Size of the entire filesystem
	\item Information on free data blocks
		\begin{itemize}
		\item Number of free blocks
		\item A “list” of free blocks
		\end{itemize}
	\item Information on free inodes
		\begin{itemize}
		\item Number of free inodes
		\item A “list” of free inodes
		\item The inode for the root (“/”) directory (inode 128 = root)
		\end{itemize}
	\end{itemize}
	
\item Linux/Unix File Allocation
	\begin{itemize}
  	\item One inode per file / directory
  	\item File block addresses are stored in inode and also in indirect blocks
  	\item N (10 or 12) direct pointers and 3 indirect pointers
  		\begin{itemize}
  		\item N direct pointers hold the addresses of the first N data blocks used by a file
  		\item The N+1st pointer holds the address of a single indirect block
  		\item The N+2nd pointer holds the address of a double indirect block
  		\item The N+3rd pointer holds the address of a triple indirect block
  		\end{itemize}
  	\end{itemize}
  		
\item Indirect Blocks
	\begin{itemize}
  	\item Indirect blocks are data blocks used for storing pointers to data blocks
  	\item Single indirect blocks
  		\begin{itemize}
  		\item Contain pointers to data blocks
  		\end{itemize}
  	\item Double indirect blocks
  		\begin{itemize}
  		\item Contain pointers to single indirect blocks
  		\end{itemize}
  	\item Triple indirect blocks
  		\begin{itemize}
  		\item Contain pointers to double indirect blocks
  		\end{itemize}
  	\end{itemize}
  		
\item Free (Disk) Block Management
	\begin{itemize}
  	\item Bit Vector/Bitmap
  	\item “Linked-List” (Free List)
  		\begin{itemize}
  		\item Simple
  		\item Grouping
  		\item Counting
  		\end{itemize}
  	\item In the “list” can be implemented as a tree (for faster search)
  	\end{itemize}
  	
\item Bit Vector / Bitmap
	\begin{itemize}
  	\item Each block is represented by 1-bit (0 = free, 1= used)
  	\item Number of bytes for bitmap = ⅛ number of total disk blocks
  		\begin{itemize}
  		\item A disk of 64 blocks => 64 bits (8 bytes) needed for bitmap
  		\end{itemize}
  	\item Fast to locate free blocks: use bitwise operations
  	\item Require extra space to maintain free blocks (tooooo much wasted space)
  		\begin{itemize}
  		\item A disk with capacity 512G (= 239) and block size of 512 bytes (= 29
  		\item Number of blocks = 239/29 = 230 blocks
  		\item Size of bitmaps 230/23 = 227 bytes (128M) of bitmap
  		\item Percentage of “wasted space” = 227 / 239 = 2-12 = 1/4096 ≈ 2.5%
  		\end{itemize}
  	\end{itemize}
  		
\item Linked-List of Free Blocks / Free List
	\begin{itemize}
  	\item Improvements over Bitmap technique
  		\begin{itemize}
  		\item Chain all the free blocks together
  		\item The superblock keeps the “head” and “tail” of this list
  		\end{itemize}
  	\item Place the “next” pointer inside (at the end) of the freeblocks
  	\item Searching for contiguous free blocks requires reading each block in the chain (too many disk I/O)
  	\end{itemize}
  	
\item Efficiency of Disk Block Operations
	\begin{itemize}
  	\item Any bytes in RAM can be accessed within the same amount of time (ns)
  	\item But, blocks on a disk may have different access time depending on their relative distance to the R/W head current position
  	\item Updating just one byte on a disk block requires writing the entire block
  	\item Design objectives of various FS are
  		\begin{itemize}
  		\item higher storage capacity (how to support Petabytes or more)
  		\item fewer disk operations
  		\item highly consistent data (fewer recovery required)
  		\end{itemize}
  	\end{itemize}

\item Critical On-Disk Data Structures
	\begin{itemize}
  	\item A filesystem essentially holds (at least) three separate trees:
  		\begin{itemize}
  		\item A tree of files and directories (user-owned)
  		\item A tree of free data blocks (maintained by OS)
  		\item A tree of free index blocks (maintained by OS)
  		\end{itemize}
  	\item These trees originate from the superblock!
  	\item Any updates to the FS must guarantee the three trees are in sync!
  	\end{itemize}
	
\item Low-Level Filesystem Operations
	\begin{itemize}
  	\item To write a file into a (Linux) filesystem
  		\begin{itemize}
  		\item Write the file contents into the data blocks
  		\item Update the inode (file size, timestamps, block pointers, ….)
  		\item Update the superblock (update the free-block list & free-inode list)
  		\end{itemize}
  	\item When the above sequence does not run to completion (i.e. by power failure), the filesystem records only a partial (inconsistent) data/metadata of your file
  	\end{itemize}

\item When a new file is created
	\begin{itemize}
  	\item Its data blocks are allocated ⇒ [superblock update]
  	\item One inode is allocated ⇒ [superblock update and inode update]
  	\item The directory entry for the file is update ⇒ [data block update]
  	\item The inode of the containing directory is updated ⇒ [inode update
  	\end{itemize}
  	
\item Database Transactions & (Intent) Logs
	\begin{itemize}
  	\item To guarantee data integrity
  	\item Multiple operations that update different parts of the DB are performed under one transaction: “BEGIN TRANSACTION” and “COMMIT”/”ROLLBACK”
  		\begin{itemize}
  		\item In addition to changing the data, a transaction also logs the intended changes (insert, delete, update)
  		\item The log must be securely saved PRIOR TO the modification of the actual data
  		\end{itemize}
  	\item At recovery time, the contents of the log are compared with the actual data and unfinished transactions can be recovered
  	\end{itemize}

\item Journaling / Log-Structured FS
	\begin{itemize}
  	\item Every modification to the FS must be associated with a log entry
  		\begin{itemize}
  		\item Log entry format: timestamp, action, blockid, oldvalue, newValue
  		\item Log entries are created for ALL types of modification (including inode and superblock)
  		\end{itemize}
  	\item The log must be saved FIRST before the actual block updates take place
  	\item Delete the log after the block updates are successful
  	\item Recovery after system crash: use the existing log entries to restore FS
  	\end{itemize}

\end{enumerate}

\end{document}
